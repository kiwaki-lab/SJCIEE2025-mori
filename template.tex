%%%%%%%%%%%%%%%%%%%%%%%%%%%%%%%%%%%%%%%%%%%%%%%%%%%%%%%%%%%%%%%%%%%%%%%%%
% SJCIEE manuscript guide/template
% version 1.1, May 27, 2022. by T. Ueta
%%%%%%%%%%%%%%%%%%%%%%%%%%%%%%%%%%%%%%%%%%%%%%%%%%%%%%%%%%%%%%%%%%%%%%%%%
\documentclass[10pt,twocolumn]{jsarticle}            % for platex 
%\documentclass[10pt,twocolumn]{ltjsarticle}         % for lualatex 
\usepackage[top=30mm, bottom=25mm, left=18mm, right=18mm]{geometry}
\usepackage[bookmarks=false, colorlinks=true, 
	dvipdfmx,                        % comment this line out for lualatex 
	linkcolor=black, urlcolor=black, citecolor=black ]{hyperref}
\usepackage{newtxtext,newtxmath}
\usepackage{wrapfig}
\usepackage[dvipdfmx]{graphicx,xcolor}               % for platex
%\usepackage{graphicx,xcolor}                        % for lualatex 
\usepackage{secdot}

\sectiondot{subsection}
\columnsep=12pt \columnseprule=0pt \jot=0pt
\topsep=0pt \parskip=0pt \parindent=12pt
\floatsep=10pt \textfloatsep=10pt \intextsep=10pt
\setlength{\abovedisplayskip}{10pt plus 5pt}
\setlength{\belowdisplayskip}{8pt plus 5pt}
\renewcommand{\textfraction}{0.1}
\renewcommand{\floatpagefraction}{0.1}
\renewcommand{\topfraction}{1.0}
\renewcommand{\baselinestretch}{0.9}
\makeatletter
\def\section{\@startsection {section}{1}{\z@}%
{1.0ex plus 0.5ex minus .2ex}{0.5ex plus .2ex}{\large\bfseries}}
\def\subsection{\@startsection{subsection}{2}{\z@}%
{0.7ex plus 0.3ex minus .2ex}{0.3ex plus .2ex}{\normalsize\bfseries}}
\def\subsubsection{\@startsection{subsubsection}{3}{\z@}%
{0.4ex plus 0.2ex minus .2ex}{0.2ex plus .2ex}{\normalsize}}
\renewenvironment{thebibliography}[1]{%
	\subsection*{\refname}\@mkboth{\refname}{\refname}%
	\list{\@biblabel{\@arabic\c@enumiv}}%
		{\settowidth\labelwidth{\@biblabel{#1}}%
		\leftmargin\labelwidth \advance\leftmargin\labelsep
		\setlength\baselineskip{13pt} \setlength\itemsep{0pt}
		\@openbib@code \usecounter{enumiv}%
		\let\p@enumiv\@empty \renewcommand\theenumiv{\@arabic\c@enumiv}}%
	\sloppy \clubpenalty4000 \@clubpenalty\clubpenalty
\widowpenalty4000 \sfcode`\.\@m}
{\def\@noitemerr {\@latex@warning{Empty `thebibliography' environment}}%
\endlist}
\makeatother
\pagestyle{empty}
%%%%%%%%%%%%%%%%%%%%%%%%%%%%%%%%%%%%%%%%%%%%%%%%%%%%%%%%%%%%%%%%%%%%%%%%%
% Please edit below
%%%%%%%%%%%%%%%%%%%%%%%%%%%%%%%%%%%%%%%%%%%%%%%%%%%%%%%%%%%%%%%%%%%%%%%%%
\begin{document}

\twocolumn[
\begin{center}
{\Large 
% 講演番号刷り入れ用余白を作るために\rule で空白を挿入しています.
% 刷り上がりを見ながら50mm * 45mm の空白を確保してください.
\rule{10mm}{0mm}電気電子情報関係学会四国支部連合大会原稿執筆ガイド\\
Manuscript Preparation Guide for the SJCIEE\\
}
\vspace*{3pt} 
{ \large
\begin{tabular}{ccc}
電気 太郎${}^1$ & 四国 次郎${}^2$ & 連合 三郎${}^3$ \\
T. Denki${}^1$  & J. Shikoku${}^2$ & S. Rengoh${}^3$\\
\end{tabular}\\
(電気大学${}^1$, 情報大学${}^2$, 通信高等専門学校${}^3$)
}
\end{center}
\vspace*{12pt} 
]

\section{原稿の構成}
本稿は,電気・電子・情報関係学会四国支部連合大会の原稿執筆について述べる.
以下のレイアウト,書式に従わない原稿については返戻されることがあるので注意.

\subsection{全体のレイアウト}

\begin{wrapfigure}[14]{l}{40mm}
\centering
\includegraphics[width=40mm]{layout.pdf}
\caption{余白}
\label{fig:layout}
\end{wrapfigure}

\textbf{原稿サイズ,分量:} 原稿は\textcolor{red}{A4縦置き}とし,
分量は\textcolor{red}{1ページを越えてはならない}.

\textbf{空間・余白:}講演番号等を挿入するため,
\textcolor{red}{左上端に50 mm $\times$ 45 mm の空間を設けること}.
上下左右の余白は図\ref{fig:layout}のとおり.空間・余白内には
ヘッダやフッタ,ページ番号などは入れないこと.

\textbf{タイトル:}
タイトルは本文より大きい文字を用い,上記空間を避けて記述する.
日本語論文の場合は必ず英語タイトルを,日本語タイトルの下に併記する.

\textbf{著者名:}
タイトルの下に書く.日本語論文の場合は必ずローマ字の著者名を併記する.
その書式は``名のイニシャル'' + ピリオド+ ``姓のフルスペル''とする.

\textbf{所属:}
著者が一人の場合は,``氏名(所属)''のように,著者名に続けて所属を書く.
著者が複数人の場合は,氏名を書き,行を改めて所属を書く.このとき,
全ての所属を一組の全角括弧``( )''で囲み,
右肩上付きのアラビア数字でそれぞれの所属を対応させる.

\subsection{本文書式}
\begin{itemize}
\itemsep=0pt
\item \textcolor{red}{本文は二段組とし,10 pt のフォントを使用すること.}
書体は,和文であれば明朝体,英文であれば Times Roman系フォントを使用し,
見出し等は和文であればゴシック体,英文であれば bold faceを用いる.
MS Wordを使用する場合,英文について等幅フォントが適用されないよう
留意すること.
\item 和文句点は「.」読点は「,」で,それぞれ全角で記述する.
\item 全角英数字,半角カタカナは,
特に使用する必要がある場合以外は避けること.
\end{itemize}

\subsection{参考文献書式} 
文献は,本文中で\cite{shikoku} のように出現順に番号で参照し,
論文末尾に参考文献リストを構成する.
また,リストの見出し「参考文献」には節番号を振らない.
リスト記述例は本稿末尾を参照のこと.

\section{PDF作成}
\label{sec:pdf}
投稿された原稿は,論文番号,ページ番号,会議名称などが
PDFレベルで追加加工された上で,論文集に収録される.
PDFのスクリーン上の見栄え,印刷刷り上がり品質について,
以下を参照のうえ,著者自身で調整すること.

\subsection{ファイルサイズ}
論文集配布の都合上,
\textcolor{red}{原稿のファイルサイズ上限は1MByte 以内とする.}
\ref{sec:images}節にも関係するが,上限を超過するようであれば,
データの適切な間引きや,画像品質調整を検討すること.

\subsection{フォントの埋め込み}
原稿内の全てのフォントについて確実に埋め込むこと.
たとえば Acrobat では ファイル→ プロパティ→ フォントと進み,
全てのフォントについて「埋め込みサブセット」となっているかを確認する.
埋め込みができていないフォントについては,
PDF閲覧・印刷環境によっては文字化け,文字抜けが発生する可能性がある.
特に図表内で用いられたテキストについて注意のこと.

\subsection{画像等の解像度}
\label{sec:images}
画像等は600 dpi 以内で高品質な仕上がりになるよう調整する.
A4刷り上がりで十分視認性を確保すること.
アンチエイリアシングやジャギーの発生に注意すること.

\subsection{セキュリティ設定とハイパーテキスト}
\textcolor{red}{PDF ファイルにはセキュリティを設定しないこと.}
また,ハイパーテキスト,しおり,メタデータは,
\ref{sec:pdf}節冒頭に述べた加工により削除・毀損されることがあり,
加工後のPDFにおいてはそれらの完全性は保証されない.

\begin{thebibliography}{9}
\bibitem{shikoku}
電気 太郎, 四国 次郎,
``DC/DC コンバータを用いた形態素解析の一考察,''
電気情報未来研究論文誌, No. 4, pp. 125--128, Nov. 2020. 
%
\bibitem{SJCIEE} 
S. Jaycie and I. Eyeyeah, 
\textsl{Beyond the DX}, Virtual Publisher, New York, 2019.
%
\bibitem{web}
電気・電子・情報関係学会四国支部連合大会, 
\url{https://sjciee.org} (2021年5月15日 参照)
\end{thebibliography}

\end{document}
